\documentclass[12pt]{article}
\usepackage{lingmacros}
\usepackage{tree-dvips}
\usepackage[english]{babel}
\usepackage{amsthm}
\usepackage{hyperref}
\usepackage{amsmath}
\newcommand{\N}{\mathbb{N}}
\newtheorem{theorem}{Theorem}[section]
\newtheorem{lemma}[theorem]{Lemma}
\newtheorem{claim}[theorem]{claim}
\usepackage[left=1in, right=1in, top=1in, bottom=1in]{geometry}
\usepackage{amsmath}
\usepackage{amssymb}
\usepackage{tikz}
\usepackage{mathtools}
\usetikzlibrary{automata,positioning,arrows}

\begin{document}
\begin{center}
\section*{ Assignment 2  Computational Models
 - Spring 2022}
\end{center}
\subsection*{Exercise 1}
 Define the language $L_n$ over alphabet $\sum_n = \lbrace 1, 2, ..., n\rbrace$ to be the set of words
which do not contain all the letters from $\sum_n$, now lets describe DFA $M=(Q,\sum ,\delta,q_0,F)$ s.t M accept $L_n$.
 \[
Q=\lbrace q_k | k \subseteq \lbrace \Sigma_n \rbrace \rbrace,\Sigma =\Sigma_n,q_0={q_{\phi}},F=\lbrace Q/\lbrace 1,2,3\dots n \rbrace \rbrace
\]
\[
\delta(q_k,\sigma) = 
     \begin{cases} 
      q_{k\cup \sigma } &\quad\text{if } \sigma \not\in \lbrace k \rbrace\\
       q_k &\quad\text{if     } \sigma \in \lbrace k \rbrace \\
     \end{cases}     
\]
its sufficient to see that the following claim hold to proof $M$ accept $L_n$
\begin{claim}
the DFA M \textbf{not} accept w iff $\lbrace \sigma: \sigma \in w \rbrace =\lbrace k_n \rbrace$ when $k_n$ is the set of all the letters from $\Sigma_n$
\end{claim}
\begin{proof}
First lets notice that Q contain all the subset from $\Sigma_n$ including the empty set witch correspond the empty world i.e $\lbrace \sigma \rbrace$ so in total we looking at $O(2^n)$ state for finite alphabet size $n$.\\
$\Rightarrow w\notin L_m$ if $w\in \Sigma _n^*$ since M have only one  state  lets call it $q_n$ which is not accepting state, hence $\hat{\delta}(q_0,w)=q_n$ we must go through at least n state to achieve $q_n$.  \\now using reduction on the number of district steps we done for $\hat{w}=\phi$ $\delta(q_0,\phi)=q_0$ and for some $\hat{w},\hat{\sigma}$ s.t\[ \hat{\sigma}\notin|\lbrace \sigma :\sigma \in \hat{w}\rbrace |=|k| , \hat{\delta}(q_0,\hat{w})=q_{k}\Rightarrow \delta(q_k,\hat{\sigma})=q_{k\cup \hat{\sigma}}=\hat{\delta}(q_0,(\hat{w}||\hat{\sigma})), \lbrace \sigma :\sigma \in \hat{w}||\hat{\sigma}\rbrace |=|k+1| \]
 hence we done n district steps so $| \lbrace \sigma :\sigma \in w\rbrace|=n$
 since M is DAG (except the self loops) w contain all the letters from the alphabet\\
 $\Leftarrow$ $w:= \forall \sigma \in \Sigma_n   |\sigma \in \lbrace w \rbrace $ now lets look at the first $\sigma \in w$, so $\delta(q_0,\sigma)=q_{\sigma}\in Q$ if the next letter in w hold $\hat{\sigma}\neq \sigma ,\delta(q_{\sigma},\hat{\sigma})=q_{\sigma}$ but we know that w have n district letters so in total  we get
\[
\hat{\delta}(q_0,w)=\hat{\delta}(q_{\sigma_1},\underbrace{\hat{w}}_{\lbrace \hat{w} \rbrace=\lbrace w/ \sigma_1\rbrace })=\hat{\delta}(q_{\lbrace \sigma_1,\sigma_2\dots \sigma_k \rbrace},\underbrace{\hat{w}}_{\lbrace \hat{w} \rbrace=\lbrace w/ \sigma_1,\sigma_2\dots \sigma_k \rbrace })=\delta(q_{\lbrace w/ \sigma_n \rbrace},\sigma_n)=q_n \notin F
\]  
hence $w\notin L_m$.

\end{proof}
\subsection*{Exercise 2}
When A,B regular languages over same alphabet. $A \diamond B = \lbrace xy : x \in A \wedge y \in B \wedge |x| = |y|\rbrace$ lets define 
\[
\Sigma=\lbrace 0,1 \rbrace, A=L(0^*), B=L(1^*)
\]
by assuming that $ A \diamond B $ is regular language, using the property of regular language operation we get.
 \[(A \diamond B )\cap \lbrace 0^*,1^*\rbrace=\lbrace 0^i,1^i|i\geq 0 \rbrace
 \]
which lead to contradiction since $\lbrace 0^i,1^i|i\geq 0 \rbrace $ is not regular language
\subsection*{Exercise 3}
$\text{\textbf{(a)} } L_1 = \lbrace w : \# a(w) \geq \# b(w)\rbrace  \text{ over } \Sigma =  \lbrace a, b, c\rbrace$\\
  $L_1$ is not regular while assuming it is. for fix $\ell$ lets $w=b^\ell a^\ell$ ,$w\in L_1$ based on Pumping Lemma we can notice $|w|>\ell$, and for any patrion of $w=xyz$ such that $|y|>0,|xy|\leq \ell$ hence y is in the form  of $b^k$ now lets look at 
  \[
  b^{\ell+k}a^{\ell}\Rightarrow \#_{b}(b^{\ell+k}a^{\ell})>\#_{a}(b^{\ell+k}a^{\ell})\Rightarrow b^{\ell+k}a^{\ell} \notin L_1
  \]
we got an contradiction hence $L_1$ is not regular\\\\
$\text{\textbf{(b)} } L_2 = \lbrace w : |w| \in \N_{even} \wedge w=w^R )\rbrace  \text{ over } \Sigma =  \lbrace 0,1 \rbrace$\\
$L_2$ is not regular while assuming it is. for fix $\ell$ lets $w=1^\ell001^\ell$ ,$w\in L_2$ based on Pumping Lemma we can notice $|w|>\ell$, and for any patrion of $w=xyz$ such that $|y|>0,|xy|\leq \ell$ hence y is in the form  of $1^k$ now lets look at
  \[
  (1^{k+\ell}001^\ell)\neq (1^{k+\ell}001^\ell)^R\Rightarrow 1^{k+\ell}001^\ell \notin L_2
  \]
we got an contradiction hence $L_2$ is not regular\\\\
$\text{\textbf{(c)} } L_3 = \lbrace w : |w| \in \N_{even} \wedge w=w^R )\rbrace  \text{ over } \Sigma =  \lbrace 0 \rbrace$\\ 
$L_3$ is regular language since we can be written  as regular expression
\[
L_3=L((00)^*)=\lbrace 00\rbrace^*=\lbrace\epsilon ,00,0000,\dots\rbrace
\]\\
$\text{\textbf{(d)} } L_4 = \lbrace w : |w| \in \N \text{ s.t } |w|=n^3 \rbrace  \text{ over } \Sigma =  \lbrace 0,1 \rbrace$\\ 
  $L_4$ is not regular while assuming it is. for fix $\ell$ lets $w=0^{\ell^3} $ ,$w\in L_4$ based on Pumping Lemma we can notice $|xyz|>\ell$, and for any patrion of $w=xyz$ such that $|y|>0,|xy|\leq \ell$. hence for some k,m such that $k+m\leq \ell$ this means that :
  \[x=0^k,y=0^m,z=0^{\ell^3-k-m}\Rightarrow,xyz=0^{\ell^3}
  \] 
by  our assumption for any $n\in \N,$ $ xy^nz=0^{\ell^3+m(n-1)}\in L_4$, lets choose n=2  \[\text{since $1\geq m \geq \ell$ we will get for some $t$}
\Rightarrow \underline{t^3=\ell^3+m }
\]
\[\text{but } t^3\geq(\ell+1)^3=\ell^3+3\ell^2+3\ell+1>
\ell^3+\ell\geq\ell^3+m\Rightarrow \underline{t^3>\ell^3+m}
\]
we got an contradiction hence $L_4$ is not regular
\subsection*{Exercise 4}
\begin{claim}
For language L s.t L satisfies pumping lemma with pumping constant $\ell$
, \\$L||L$ satisfies pumping lemma with pumping constant  $2\ell$
\end{claim}
\begin{proof} Lets $w_1,w_2\in L'$ i.e $w_1\in L \wedge w_2 \in L $ and lets assume $|w_1w_2|\geq 2\ell$ to imply the pumping lemma, now we can notice that there is 2 possible scenario  
\begin{itemize}
  \item $|w_1|\geq \ell$ hence we can write $w_1$ as $w_1=xyz  \Rightarrow$ and we can write $w_1w_2$ as $w_1w_2=xyz||w_2$ which stand with the lemma property  
  \item $|w_1|<\ell$ so $|w_2|>\ell$ can write $w_2$ as $w_2=xyz$ and $|xy|\leq\ell\Rightarrow |w_1xy|\leq 2\ell,|y|>0$  we can write $w_1w_2$ as $\underbrace{w_1x}_{x'}yz$ which stand with the claim property   
\end{itemize}
\end{proof}
\subsection*{Exercise 5}
\subsection*{(a)}
Lets L be a regular language over alphabet $\Sigma$. and we will proof the language $L'$ define $L'=\lbrace xyz:xy^Rz \in L \rbrace$ is regular. since L is regular $\Rightarrow$ exists some DFA that accept L $M=(Q,\Sigma ,\delta,q_0,F)$ now lets define few new DFA's  s.t:
\begin{itemize}
\item $M_{q_0,q_k}=(Q,\Sigma ,\delta,q_0,q_k)$ for any $q_k\in Q $ this DFA will cover any path that accessible from $q_0$ 
\item now lets look at $M_{q_k,q_j}=(Q,\sum ,\delta,q_k,q_j)$lets the language of $M_{q_k,q_j}$ is define by $L(M_{q_k,q_j})=\lbrace w :\exists q_k,q_j \in Q | \text{ exists path s.t }q_k \rightsquigarrow q_j \rbrace$ ,since $L(M_{q_k,q_j})$ is regular $\Rightarrow$ $rev(L(M_{q_k,q_j}))$ is regular (Recitation 3  ex.1). for any $q_k,q_j\in Q$ lets define the following language  $L(M_{q_k,q_j})^R$ 
\item $M_{q_k,F}=(Q,\Sigma ,\delta,q_k,F)$ for any $q_k\in Q $ this DFA will cover any path that end in $F$
\end{itemize}
\begin{claim}
$L'=\cup_{q_i,q_j}L(M_{q_0,q_i})||L(M_{q_i,q_j})^R||L(M_{q_j,F})$
\end{claim}
\begin{proof}
\[ w=xyz \in L' \Leftrightarrow xy^Rz \in L 
\Leftrightarrow \exists \hat{\delta}(\hat{\delta}(\hat{\delta}(q_0,x),y^R),z) \in F_m \Leftrightarrow
\]
\[
\exists M_{q_0,q_i}:\hat{\delta}(q_0,x)\in F_{ M_{q_0,q_i}} \wedge \exists L(M_{q_i,q_j})^R : y\in L(M_{q_i,q_j})^R \wedge \exists M_{q_j,F}:\hat{\delta}(q_j,z)\in F_{ M_{q_j,F}}=F_m
\]
\[ \Leftrightarrow xyz\in \cup_{q_i,q_j}L(M_{q_0,q_i})||L(M_{q_i,q_j})^R||L(M_{q_j,F})
\]
\end{proof}
\subsection*{(b)}
Lets L be a regular language over alphabet $\Sigma$. and we will proof the language $L'$ define $L'=\lbrace xy  \in \Sigma^* :(x\in L) \text{ XOR } (y\in L) \rbrace$ is regular.
using the  closure property of regular language
\begin{itemize}
\item $L=\lbrace x\in \Sigma^* : x\in L \rbrace$ and $\overline{L}=\lbrace x\in \Sigma^* : x\notin  L \rbrace$ are regular.
\item $L||\overline{L}=\lbrace xy\in \Sigma^* : x\in L \wedge y \notin L \rbrace$ and $\overline{L}||L=\lbrace xy\in \Sigma^* : x\notin L \wedge y \in L \rbrace$ are regular.
\item $(L||\overline{L}) \cup (\overline{L}||L)=\lbrace xy\in \Sigma^* : (x\in L \wedge y\notin L )\vee (x\notin L \wedge y\in L ) =(x\in L) \text{ XOR } (y\in L) \rbrace$ is regular
\end{itemize}
\subsection*{Exercise 6}
$\text{\textbf{(a)} }\lbrace w \in \Sigma^*: \#_0 (w) \leq 3 \rbrace $\\
lets express the following as regular expression 
\[
 \underbrace{1^*}_{I}\cup \underbrace{1^*01^*}_{II} \cup \underbrace{1^*01^*01^*}_{III}\cup \underbrace{1^*01^*01^*01^*}_{IV} 
\]
(I) is the regular expression of the language that contain non zeros at all.
\\ (II) is the regular expression of the language that contain exactly 1 zero.
\\ (III) is the regular expression of the language that contain exactly 2 zero.
\\ (IV) is the regular expression of the language that contain exactly 3 zero.\\
hence combine them all together will hold regular expression of the language that contain at most 3 zeros.
\\\\$\text{\textbf{(b)} }\lbrace w \in \Sigma^*:|w| \text{ mod 4}=2 \rbrace $\\
lets express the following as regular expression 
\[
 \underbrace{(0 \cup 1 )(0\cup 1) }_{(I)}\underbrace{((0\cup 1)(0\cup 1)(0\cup 1)(0\cup 1))^*}_{II}
\]
(I) is the regular expression of the language that  size is exactly 2. 
\\(II) is the regular expression of the language from size 0,4,8.. \\
hence the concatenate of (I) and (II) will hold the $|w|$ mod $4=2$ property
\end{document}